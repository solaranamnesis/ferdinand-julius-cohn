\documentclass[a4paper, 11pt, oneside, english]{article}
\usepackage{kmath, kerkis}
\usepackage[T1]{fontenc}

% Load encoding definitions (after font package)

\usepackage{textalpha}

\usepackage{listings}
\lstset{basicstyle=\ttfamily}

% Babel package:
\usepackage[german]{babel}

% With XeTeX$\$LuaTeX, load fontspec after babel to use Unicode
% fonts for Latin script and LGR for Greek:
\ifdefined\luatexversion \usepackage{fontspec}\fi
\ifdefined\XeTeXrevision \usepackage{fontspec}\fi

% ``Lipsiakos"' italic font `cbleipzig`:
\newcommand*{\lishape}{\fontencoding{LGR}\fontfamily{cmr}%
		       \fontshape{li}\selectfont}
\DeclareTextFontCommand{\textli}{\lishape}

\usepackage{booktabs}
\usepackage{graphicx}
\setlength{\emergencystretch}{15pt}
\graphicspath{ {./ } }
\usepackage[figurename=]{caption}
\usepackage{float}
\usepackage{fancyhdr}
\usepackage{microtype}
\begin{document}
\begin{titlepage} % Suppresses headers and footers on the title page
	\centering % Centre everything on the title page
	\scshape % Use small caps for all text on the title page

	%------------------------------------------------
	%	Title
	%------------------------------------------------
	
	\rule{\textwidth}{1.6pt}\vspace*{-\baselineskip}\vspace*{2pt} % Thick horizontal rule
	\rule{\textwidth}{0.4pt} % Thin horizontal rule
	
	\vspace{0.75\baselineskip} % Whitespace above the title

        {\LARGE Über Bakterien, \\\large die kleinsten lebenden Wesen} % Title
	
	\vspace{0.75\baselineskip} % Whitespace below the title
	
	\rule{\textwidth}{0.4pt}\vspace*{-\baselineskip}\vspace{3.2pt} % Thin horizontal rule
	\rule{\textwidth}{1.6pt} % Thick horizontal rule
	
	\vspace{1\baselineskip} % Whitespace after the title block
	
	%------------------------------------------------
	%	Subtitle
	%------------------------------------------------
	
	{Von \\\Large Dr. Ferdinand Cohn\\\large Professor an der Universität zu Breslau} % Subtitle or further description
	
	\vspace*{1\baselineskip} % Whitespace under the subtitle

        {\normalsize Mit Holzschnitten.}

         \vspace*{1\baselineskip} % Whitespace under the subtitle

	%------------------------------------------------
	%	Editor(s)
	%------------------------------------------------
	
        %------------------------------------------------
	%	Cover photo
	%------------------------------------------------
		
	%------------------------------------------------
	%	Publisher
	%------------------------------------------------
		
	\vspace*{\fill}% Whitespace under the publisher logo

	% Publication year

	{Berlin, 1872} % Publisher
 
        {\small C. G. Lüderitz'sche Verlagsbuchhandlung, Carl Habel}

	\vspace{1\baselineskip} % Whitespace under the publisher logo

    Internet Archive Online Edition  % Publication year
	
	{\small Namensnennung Nicht-kommerziell Weitergabe unter gleichen Bedingungen 4.0 International } % Publisher
\end{titlepage}
\clearpage
\setlength{\parskip}{1mm plus1mm minus1mm}
\paragraph{}
Im Jahre 1875 feiert die Wissenschaft das zweihundertjährige Jubiläum der Entdeckung einer neuen Welt durch Anton Leeuwenhoek. Ohne gelehrte Bildung, aber mit lebhaftem Forschertrieb ausgestattet, wie ihn das siebzehnte Jahrhundert, das Zeitalter der größten naturwissenschaftlichen Entdeckungen in so vielen begabten Geistern anregte, hatte Leeuwenhoek schon als Jüngling den Kaufmannsladen von Amsterdam verlassen, in den er als Lehrling eingetreten, und sich in seiner Heimat Delft mit dem bescheidenen Posten eines Beschließers der Schöppenstube begnügt, den er durch 39 Jahre verwaltete; seine Muße aber und sein großes mechanisches Talent verwendete er zur Anfertigung von Vergrößerungsgläsern, mit denen er anfänglich nach Dilettantenart Mückenflügel und Bienenstachel, Schmetterlingsschuppen und Moospflänzchen beobachtete; aber die bis dahin unerreichte Vollkommenheit seiner Mikroskope und seine klare und ausdauernde Beobachtungsgabe enthüllten ihm bald "`verborgene Naturgeheimnisse"'\footnote{Leeuwenhoek, \emph{Arcana naturae detecta}.} die er in begeisterten Briefen der Königlichen Gesellschaft der Wissenschaften in London mitteilte. Im April 1675 kam Leeuwenhoek auf den Einfall, ein Glasröhrchen voll stehenden Regenwassers unter eines seiner Mikroskope zu bringen; mit staunender Bewunderung erblickte er im Wasser wunderliche Gestalten, Glöckchen, die sich aufblähten und zusammenzogen, Kügelchen, die lebhaft hin und herschossen; im ersten Augenblick glaubte er die lebendigen Atome zu erblicken, aus denen nach der Philosophie des alten Demokrit alle Körper bestehen, und aus deren Wirbelbewegungen sein Zeitgenosse Descartes von Neuem die Welt sich aufbauen ließ. Bald aber überzeugte sich Leeuwenhoek, dass er es mit Tierchen (\emph{animalcula}) zu thun habe, die dem bloßen Auge unsichtbar, in zahlreichen Formen den Wassertropfen beleben; sie wurden später besonders reichlich in Aufgüssen von Pfeffer, Heu und anderen Tier- und Pflanzenstoffen gefunden, und erhielten deshalb den Namen der Aufguss- oder Infusionstierchen (\emph{Infusoria}). Gerade ein Jahrhundert nach Leeuwenhoek fand sich ein Forscher in Dänemark, der 12 Jahre seines Lebens auf die Beobachtung dieser kleinsten Tiere verwendete, von denen er in den süßen und Meergewässern von Kopenhagen an 380 verschiedene Arten benannte und abbildete.\footnote{O. F. Müller, \emph{Vermium terrestrium et fluviatilium historia} 1774. \emph{Animalcula infusoria} 1786.} Im letzten Jahrhundert mehrte sich in raschem Verhältnis die Zahl der Naturforscher, welche mit immer vollkommeneren Instrumenten in die unsichtbare Welt einzudringen suchten; außer den zahlreichen Tiergeschlechtern wurde auch eine ganz eigentümliche mikroskopische Flora entdeckt, deren Gestaltung und Entwickelung durchaus verschieden ist von den sichtbaren Gewächsen. War Leeuwenhoek der Columbus dieser neuen Welt, so können wir Ehrenberg\footnote{Von dem Griechischen \emph{Bakterion}, Stäbchen.} als den Humboldt derselben bezeichnen; denn seit dem Jahre 1829 bis auf den heutigen Tag hat Ehrenberg mit eisernem Fleiße deren verborgene Gebiete bis an die äußersten Grenzen durchforscht, und nicht bloß die mikroskopischen Wesen gründlicher und getreuer als seine Vorgänger beschrieben, abgebildet und geordnet, sondern auch die ungeahnte Bedeutung enthüllt, welche den Geschöpfen der unsichtbaren Welt in der gesamten Naturordnung zukommt, nicht bloß in der Gegenwart, sondern auch in früheren geologischen Zeitaltern.

Jedermann weiß, in wie verschiedenen Größenverhältnissen das Leben der sichtbaren Welt sich verkörpert. Zu den kleinsten Tieren, die das unbewaffnete Auge noch unterscheidet, gehören die Milben, die im Käse oder auf zuckerreichen Früchten oft in unzähligen Schaaren nisten; ihre Größe verhält sich zu der des Menschen, etwa wie der Sperling zum Straßburger Münster; ähnlich mag das Verhältnis sein zwischen der Riesentanne und dem Moose, das auf ihrer Rinde wuchert. Von den Tierchen die Leeuwenhoek entdeckte, gibt derselbe an, dass ihre Größe sich zur Milbe erhalte, wie die Biene zum Gaul. Je mehr in den letzten Jahrzehnten die Mikroskope verbessert und ihre Vergrößerungskraft gesteigert wurde, desto kleinere Wesen wurden der scharfen Beobachtung zugänglich; denn unter den Tieren und Pflanzen der unsichtbaren Welt finden sich noch ähnliche Größenunterschiede, wie zwischen dem Hering und dem Walfisch.

Je kleiner aber die Wesen, desto einfacher zeigte sich ihr Bau, desto unvollkommener ihre Lebenstätigkeit, desto tiefer ihre Stellung in der Rangordnung der Geschöpfe. Unter den Tieren der mikroskopischen Welt sind nur äußerst wenige, welche die Organenfülle eines Insekts, eines Krebses, selbst eines Wurmes besitzen; die eigentlichen Infusionstierchen stehen auf der untersten Stufe des Tierreichs. Ebenso finden wir unter den mikroskopischen Pflanzen keine einzige, welche den entwickelteren Bau der blühenden Gewächse erreicht, oder auch nur der tieferen Klasse der Farne angehört; nur die niedersten Pflanzenformen, die wir gewöhnlich als Algen und Pilze bezeichnen, bilden die Wälder und Wiesen der unsichtbaren Welt.

Je mehr sich aber der innere Bau der mikroskopischen Wesen vereinfacht, desto weniger treten die Merkmale hervor, welche in der sichtbaren Welt Tiere und Pflanzen so leicht unterscheiden. Den Infusionstierchen fehlen Muskeln und Nerven; Gefäße und Atmungsorgane sind nur äußerst unvollkommen entwickelt; auf der andern Seite zeigen die mikroskopischen Pflanzen selbstständige Bewegungen, und selbst Bewegungsorgane, wie wir sie nur bei Tieren zu finden gewohnt sind. In den niedersten Wesen endlich scheint Tier und Pflanze ineinandergeflossen, und der Naturforscher gerät in Zweifel, welchem der beiden Reiche er sie zuweisen soll.

Die kleinsten aber und zugleich die allereinfachsten und niedersten aller lebenden Wesen nennen wir Bakterien\footnote{Untersuchungen über Bakterien in "`Beiträge zur Biologie der Pflanzen."' Herausgegeben von Dr. Ferdinand Cohn. Heft 2. 1872, mit einer Tafel.}; sie bilden die Grenzmark des Lebens; jenseits derselben ist nichts Lebendiges mehr vorhanden, soweit wenigstens unsere heutigen mikroskopischen Hülfsmittel reichen. Und diese sind nicht gering; die stärksten unserer Vergrößerungsgläser, die Immersionssysteme von Hartnack geben 3-4000fache Vergrößerungen; und könnte man einen Menschen unter einem solchen Linsensystem ganz überschauen, er würde so groß erscheinen, wie der Montblanc oder gar der Chimborasso. Aber selbst unter diesen kolossalen Vergrößerungen sehen die kleinsten Bakterien nicht viel größer aus, als die Punkte und Kommas eines guten Drucks; von ihren inneren Teilen ist wenig oder gar Nichts zu unterscheiden, und selbst die Existenz würde von den meisten verborgen bleiben, wenn sie nicht in unendlichen Mengen gesellig lebten. Diese kleinsten Bakterien verhalten sich ihrer Größe nach zum Menschen, etwa wie ein Sandkorn zum Montblanc.

Ist es nun schon an und für sich wichtig, die kleinsten zugleich und die einfachsten aller lebenden Wesen genauer kennen zu lernen, so steigert sich unser Interesse an denselben durch die Erkenntnis, dass gerade diese kleinsten Wesen von der allergrößten Bedeutung sind, dass sie mit unsichtbarer, doch unwiderstehlicher Gewalt die wichtigsten Vorgänge der lebendigen und leblosen Natur beherrschen und selbst in das Dasein des Menschen zugleich geheimnis- und verhängnisvoll eingreifen.

Die Gestalt der Bakterien gleicht bald einer Kugel, oder einem Ei, bald einem kurzen oder längeren Stäbchen oder Faden, bald einem Korkzieher oder einer Schraube. Ihr Körper besteht aus einer meist farblosen eiweißartigen Substanz, in der starkglänzende Fettkörnchen eingelagert sind, und die von einer dünnen, in Kali unlöslichen Haut eingeschlossen ist. Nach der Gestalt können wir Kugel- Stäbchen- Faden- und Schraubenbakterien unterscheiden; nach der Sprache der Wissenschaft werden die Bakterien in Gattungen und Arten verteilt; der Verfasser dieses Aufsatzes hat in seiner neuesten Bearbeitung der Bakterien\footnote{Nur bei den größten Spirillen (Fig. 5) sind neuerdings bewegliche Geißeln entdeckt worden, welche Wirbel im Wasser erregen und bei den Bewegungen mittätig sind.} 6 Gattungen unterschieden, die kugeligen und eirunden als \emph{Micrococcus}, (Fig. 1.) die kurzen Stäbchen als \emph{Bacterium}, (Fig. 2.) die geraden Fädchen als \emph{Bacillus}, (Fig. 3.) die wellig gelockten als \emph{Vibrio}, (Fig. 4.) die kurzen steifen Schrauben als \emph{Spirillum}, (Fig. 5.) endlich die langen biegsamen Spiralen als \emph{Spirochaete} (Fig. 6.) bezeichnet.

\begin{figure}[H]
\centering
\includegraphics[width=0.2\textwidth,keepaspectratio]{figs/fig01.png}
\caption{\emph{Micrococcus.}}
\end{figure}

\begin{figure}[H]
\centering
\includegraphics[width=0.2\textwidth,keepaspectratio]{figs/fig02.png}
\caption{\emph{Bacterium Termo.}}
\end{figure}

\begin{figure}[H]
\centering
\includegraphics[width=0.2\textwidth,keepaspectratio]{figs/fig03.png}
\caption{\emph{Bacillus subtilis.}}
\end{figure}

\begin{figure}[H]
\centering
\includegraphics[width=0.2\textwidth,keepaspectratio]{figs/fig04.png}
\caption{\emph{Vibrio Rugula.}}
\end{figure}

\begin{figure}[H]
\centering
\includegraphics[height=0.2\textheight,keepaspectratio]{figs/fig05.png}
\caption{\emph{Spirillum volutans.}}
\end{figure}

\begin{figure}[H]
\centering
\includegraphics[height=0.2\textheight,,keepaspectratio]{figs/fig06.png}
\caption{\emph{Spirochaete plicatilis.}}
\end{figure}

\begin{figure}[H]
\centering
\includegraphics[width=0.4\textwidth,keepaspectratio]{figs/fig07.png}
\caption{Gallert von Kugel- und Stäbchenbakterien.}
\end{figure}

\paragraph{}
Fast alle Bakterien besitzen zwei verschiedene Lebenszustände, einen beweglichen und einen ruhenden. Unter gewissen Bedingungen sind sie außerordentlich lebhaft bewegt und wenn sie in dichtem Gewimmel den Wassertropfen erfüllen, bieten die nach allen Richtungen durcheinander fahrenden Bakterien einen überaus fesselnden Anblick, den man mit einem Mückenschwarm oder einem Ameisenhaufen vergleichen kann. Die Bakterien schwimmen hurtig vorwärts, dann ohne umzukehren ein Stück zurück; oder sie ziehen in Bogenlinien dahin, bald langsam zitternd und wackelnd, jetzt in plötzlichem Sprunge raketenartig fortschießend, bald darauf der Quere nachgedreht wie ein Kreisel, oder längere Zeit ruhend, um plötzlich wie der Blitz auf und davon zu fahren. Die längeren Fadenbakterien biegen ihren Körper beim Schwimmen, bald schwerfällig, bald rasch und gewandt, als bemühten sie sich durch Hindernisse ihre Bahn zu finden, wie ein Fisch, der zwischen Wasserpflanzen seinen Weg sucht. Dann stehen sie eine Zeit lang still, als müssten sie eine Weile ausruhen; plötzlich zittert der kleine Faden und schwimmt zurück, um bald darauf wieder vorwärtszusteuern. Mit all diesen Bewegungen ist stets eine rasche Achsendrehung verbunden, wie bei einer in der Mutter sich bewegenden Schraube; dies wird besonders deutlich, wenn die Stäbchen geknickt sind; dann sieht man sie gleichsam taumelnd sich umherwälzen. Wenn die wellenförmigen Vibrionen und die schraubenförmigen Spirillen sich rasch um ihre Achse drehen, so erregen sie eine eigentümliche Sinnestäuschung, als ob sie sich aalgleich schlängelten, obwohl sie völlig steif sind; oft zucken sie meteorartig hin und her, dass sie dem Beobachter kaum zum Bewusstsein kommen, oder rollen rasch durch das Gesichtsfeld; während sie jetzt an einem Ende sich festhaltend, sich mit dem andern im Kreise bewegen, gleich einer um einen Faden gedrehten Schleuder, sieht man sie bald darauf sich langsam durch das Wasser schrauben.

Fast alle älteren Beobachter haben die Bakterien als Tiere betrachtet, da ihre Bewegungen als willkürliche aufgefasst wurden. Allerdings sind es innere Lebenstätigkeiten, welche die Bewegungen der Bakterien veranlassen und die bewegende Kraft ist umso rätselhafter, als keine Bewegungsorgane sichtbar werden.\footnote{Ich nehme eine Hefezelle im Mittel als eine Kugel von 0.008 Millimeter Durchmesser, 0.00000025 Kubikmillimeter Inhalt. In der Presshefefabrik zu Gießmannsdorf bei Neiffe können täglich 100 Ztr. Presshefe gewonnen werden, die aus 75 Prozent Wasser, 25 Prozent Hefepilzen besteht.} Dennoch ist kein Zweifel, dass der Anschein der Willkür nur Täuschung, dass bei den Bakterien keine Seelentätigkeiten im Spiele sind, wie sie im Begriff der Willkür liegen, und in der Tat die Bewegungen wenigstens der höheren Tiere beherrschen. Ganz ähnliche Bewegungen, wie schon bemerkt, werden bei vielen mikroskopischen Pflanzen beobachtet, entweder andauernd, wie bei den Kieselzellen (\emph{Diatomeen}) und Schwingfaden (\emph{Oscillarien}), oder vorübergehend während der Fortpflanzung, wie bei den Schwärmsporen und Samenkörperchen der Algen und Pilze.

Die gesamte Entwickelung der Bakterien macht es in höchstem Grade wahrscheinlich, dass sie ins Pflanzenreich gehören und in die nächste Verwandtschaft der Oscillarien gehören. Auch wechselt bei den Bakterien mit dem beweglichen ein ruhender Zustand, wo sie von gewöhnlichen Pflanzenzellen sich durchaus nicht unterscheiden; sie schwärmen nur bei günstiger Temperatur, reicher Nahrung und Anwesenheit von Sauerstoff; unter ungünstigen Umständen sind sie bewegungslos; gewisse Arten, wie die Kugelbakterien und die Bakterien des Milzbrands, scheinen sich nie zu bewegen.

Wie alle lebenden Wesen, vermögen auch die Bakterien sich fortzupflanzen; diese Fortpflanzung beruht auf der Querteilung. Die Bakterie wächst, bis sie etwa das Doppelte ihrer ursprünglichen Größe erreicht hat; dann schnürt sie sich in der Mitte ein, wie eine 8, und zerbricht schließlich in ihre zwei Hälften, von denen jede in kurzer Zeit aufs Neue sich in zwei Teile teilt. Wegen des raschen Verlaufs dieser Vorgänge findet man daher die Bakterien fast immer in der Vermehrung, in der Mitte eingeschnürt oder paarweise zusammenhängend (Fig. 1-4).

Je wärmer die Luft, desto rascher verläuft die Teilung der Bakterien, desto stärker ist ihre Vermehrung; bei niederer Temperatur wird sie langsamer und hört in der Nähe des Gefrierpunktes gänzlich auf. Es verlohnt wohl der Mühe, sich durch Rechnung eine Vorstellung von der unglaublichen Massenentwickelung zu machen, deren diese kleinsten aller Wesen unter günstigen Bedingungen durch ihre Vermehrung fähig sind.

Wir nehmen an, dass eine Bakterie sich innerhalb einer Stunde in 2, diese wieder nach einer Stunde in 4, nach 3 Stunden in 8 teilen und sofort; nach 24 Stunden beträgt die Zahl der Bakterien bereits über 16 1/2 Million (16,777,220); nach 2 Tagen würde sie zu der ungeheuren Zahl von 281 1/2 Billionen, nach 3 Tagen zu 47 Trillionen anwachsen; nach einer Woche würde ihre Anzahl sich nur durch eine Ziffer von 51 Stellen ausdrücken lassen.

Um diese Zahlen leichter fasslich zu machen, wollen wir die Masse und das Gewicht berechnen, welches aus einer Bakterie in Folge ihrer Vermehrung hervorgehen kann. Die einzelnen Körperchen der gemeinsten Art der Stäbchenbakterien (\emph{Bacterium Termo}, Fig. 2) haben die Gestalt kurzer Zylinder, von 1/1000 Millimeter im Durchmesser und etwa 1/500 Millimeter Länge. Denken wir uns ein Würfelförmiges Hohlmaß von ein Millimeter Seite (ein Kubikmillimeter), so würde dasselbe nach den eben angegebenen Verhältnissen von 633 Millionen Stäbchenbakterien ohne Zwischenraum ausgefüllt werden. Nach 24 Stunden würden die aus einem einzigen Stäbchen hervorgegangenen Bakterien etwa den vierzigsten Teil eines Kubikmillimeters einnehmen; aber schon am Ende des folgenden Tages würden die Bakterien einen Raum erfüllen, der 442,570 solcher Würfel, oder was dasselbe ist, etwa 1/2 Liter oder 44 1/2 Kubikzentimetern gleich kommt. Nehmen wir den Raum, den das Weltmeer einnimmt, gleich 2/3 der Erdoberfläche, und seine Tiefe im Mittel gleich einer Meile, so ist der Gesamtinhalt des Oceans 928 Millionen Kubikmeilen; bei stetig fortschreiender Vermehrung würden die aus einem Keim entstammenden Bakterien schon nach weniger als 5 Tagen das ganze Weltmeer vollständig erfüllen; ihre Zahl würde sich dann nur durch eine Ziffer von 37 Stellen ausdrücken lassen.

Noch überraschender sind die Gewichtsverhältnisse. Setzen wir das spezifische Gewicht einer Bakterie dem des Wassers gleich, was von der Wahrheit nicht viel abweichen kann, so ergibt sich aus den oben angeführten Maßen, dass ein einziges Stäbchen 0.000,000,001,571 Milligramm, oder dass 636 Milliarden Bakterien ein Gramm, oder 636,000 Milliarden ein Kilogramm wiegen. Nach 24 Stunden würde das Gewicht der Bakterien ungefähr 1/40 Milligramm, nach 48 Stunden fast 1 Pfd. (442 Gramm) betragen, nach 3 Tagen dagegen nahezu 7 1/2 Million Kilogramm, oder ein Gewicht von 148,356 Zentnern erreichen.

Man halte solche Berechnungen nicht für müßige Spielerei; sie allein machen uns die kolossalen Arbeitsleistungen der Bakterien verständlich. Auch stützen sie sich nur auf solche Voraussetzungen, die von der Natur selbst gegeben sind; wäre z. B. die Dauer des Teilungsvorganges in Wirklichkeit auch erheblich länger als die von uns angenommene Stunde, so würden die berechneten Zahlen eben nur ein paar Stunden oder Tage später zutreffen. Wenn freilich in begrenztem Raume niemals jene Werte auch nur annähernd erreicht werden, so liegt dies nicht etwa daran, dass die Vermehrungsfähigkeit der Bakterien hinter der Rechnung zurückbleibt, sondern allein an der beschränkten Nahrung; selbstverständlich erzeugen die Bakterien den Stoff nicht selbst, der ihren Körper bildet, sondern sie nehmen ihn von außen als Nahrung auf, und es können sich daher nicht mehr Bakterien bilden, als ihnen Nahrung geboten wird. Dazu kommt, dass die übrigen Pflanzen und Tiere auf dieselben Nährstoffe angewiesen sind, und sich gegenseitig die Existenz streitig machen; jener grausame Kampf ums Dasein, der nach altem Brauch den Unterliegenden zugleich ausrottet, hält die Vermehrung der Bakterien, wie aller übrigen Wesen, in Schranken; nur wo jene die Oberhand behalten, vermögen sie sich ihrer Mitbewerber, die zugleich ihre Todfeinde sind, zu erwehren. Die Presshefefabriken geben uns aber ein anschauliches Beispiel, zu welch kolossalen Massenverhältnissen sich mikroskopische Körperchen vermehren können, wenn ihnen ausreichende Nahrung geboten, und die Konkurrenz anderer Wesen sorgfältig ferngehalten wird. Der Hefepilz übertrifft die Stäbchenbakterien in Masse und Gewicht etwa um das 160fache\footnote{Bildung von Anilinfarben aus Proteinkörpern. Journal für praktische Chemie. 1866.}; das Gewicht einer Hefezelle ist also gleich 0.00000025 Milligramm; oder 40 Millionen Hefezellen wiegen 1 Kilogramm. Werden nun in riesigen, mit geeigneter Nahrung reichlich erfüllten Bottichen die Hefezellen ungestörter Vermehrung überlassen, so können in großen Fabriken innerhalb 24 Stunden über 100 Zentner Presshefe erzeugt werden; möglicherweise sind die mehr als 50 Milliarden Zellen, die solche Masse bilden, im Verlauf eines Tages aus einem einzigen Keime hervorgegangen. ---

Wir kennen bei den Bakterien bis jetzt keine andere Vermehrung als die eben geschilderte Zweiteilung; die Erzeugung von Eiern oder Sporen, wie sie bei der Fortpflanzung aller übrigen Pflanzen und Tiere gebildet werden, ist bei diesen einfachsten Wesen noch nicht beobachtet. Nach der Teilung entfernen sich entweder die Bakterienhälften, und schwärmen als selbstständige Wesen davon; oder sie bleiben kettenartig an einander gereiht und bilden dann längere oder kürzere Fäden; in andern Fällen bleiben ganze Generationen in Kolonien, Nestern oder Ballen vereint, oder sie verbinden sich zu Haufen, welche schon dem bloßen Auge wie farblose oder auch farbige Gallert- oder Schleimmassen erscheinen, als weiße Flöckchen oder Fäden im Wasser schwimmen oder am Boden von Flüssigkeiten sich flockig absetzen. (Fig. 7.)

Die Bakterien gehören zu den am meisten verbreiteten Wesen; man kann sie geradezu allgegenwärtig nennen; sie fehlen nirgends weder in der Luft noch im Wasser; sie heften sich der Oberfläche aller festen Körper an. Aber massenhaft entwickeln sie sich nur da, wo Zersetzung und Verwesung, Gärung und Fäulnis stattfindet; bringt man ein Stückchen Fleisch, eine Erbse oder irgendeinen anderen tierischen oder Pflanzenstoff in Wasser, so wird dieses früher oder später trübe, dann milchig; es verliert seine Durchsichtigkeit, weil sich in ihm die Bakterien in den oben berechneten Verhältnissen vermehren, bis diese fast ohne Zwischenraum das Wasser erfüllen. Gleichzeitig schreitet die Fäulnis immer weiter fort, unter Entwickelung verschiedener, meist sehr übelriechender chemischer Verbindungen.

Nach einiger Zeit nimmt die Trübung ab; das Wasser wird wieder klar und geruchlos; der organische Stoff ist von den Bakterien verzehrt worden; diese hören nun auf, sich weiter zu teilen, und häufen sich am Boden unbeweglich als weißer Niederschlag an; wird neue Substanz zum Faulen zugefügt, so beginnt auch die Vermehrung der Bakterien aufs Neue.

Auch ohne Wasser in feuchter Luft vermehren sich die Bakterien, sobald sie in Zersetzung begriffene Stoffe finden; sie überziehen im dumpfigen Speiseschrank die gekochten Kartoffeln, den Käse und andere Speisen mit schleimigen, farblosen oder gefärbten Überzügen, die selbst mit bloßem Auge von den schneeweißen mit bläulichem Sporenpulver überstreuten Spinnweben der Schimmelpilze sich leicht unterscheiden; auch der weißliche Schleim, der die Zähne überzieht, wird großenteils von Bakterien gebildet.

Woher kommt es nun aber, dass sich stets Bakterien in faulenden Stoffen entwickeln? In welchem Verhältnis stehen die Bakterien zur Fäulnis? Auf diese Fragen sind verschiedene Antworten gegeben worden.

Die Einen sagen: Im Körper lebender Tiere und Pflanzen find die chemischen Elemente zu eigentümlichen, sogenannten organischen Verbindungen zusammengefügt. Der Tod löst das Band, vermittelst dessen die Lebenskraft die Elemente verknüpft; diese überlassen sich dem freien Spiel ihrer Anziehungskräfte, und ordnen sich, diesen folgend, zu neuen einfacheren Verbindungen. Gleichzeitig sucht der Sauerstoff der Luft, der zu einzelnen Stoffen des toten Körpers lebhafte Verwandtschaft besitzt, sich mit diesen zu verbinden; so entstehen Entmischungen, Zersetzungen und Neubildungen, durch welche die Form und Zusammensetzung des toten Körpers gänzlich zerstört wird; diese Vorgänge sind es, welche wir als Fäulnis und Verwesung bezeichnen; es sind rein chemische Prozesse, der Verbrennung, der Verwitterung, dem Rosten der Metalle vergleichbar. Die Bakterien finden reichliche Nahrung in den bei der Fäulnis sich bildenden Verbindungen, während sie sich von lebendigen Wesen nicht ernähren können; kein Wunder, dass ihre Keime, wenn sie auch anfänglich nur vereinzelt Zutritt gefunden, sich bei der Fäulnis so außerordentlich vermehren.

Wäre diese Auffassung richtig, so wären die Bakterien nur zufällige Begleiter der Fäulnis; es müsste Fäulnis toter Körper unter den dafür geeigneten Bedingungen auch dann eintreten, wenn die Bakterien von denselben fern gehalten werden.

Wenn wir Versuche anstellen, um die Richtigkeit dieser Vermutung zu prüfen, so ist diese Bedingung freilich nicht leicht zu erfüllen; bringen wir zum Beispiel Teile oder Säfte eines Tieres oder einer Pflanze, Fleisch, Blut, Harn, Milch oder Stücke von Blättern, Früchten, Samen in ein Glaskölbchen, so ist stets zu vermuten, dass gleichzeitig auch einige der so außerordentlich verbreiteten Bakterien mit eingeführt werden, und diese Vermutung wird fast zur Gewissheit, wenn wir in das Kölbchen noch etwas Wasser thun, da alles Wasser nachweisbar Bakterienkeime enthält. Es gibt aber ein einfaches Mittel, alle Bakterien in dem Glaskölbchen zu beseitigen; man braucht dasselbe nur eine Zeit lang zu kochen. Denn so wenig, wie irgendein anderes Tier oder Pflanze, so wenig widerstehen die Bakterien der Siedhitze; neuere Versuche haben sogar gezeigt, dass schon eine Erwärmung auf 60° C. die Bakterien tötet, nur muss diese Temperatur lange genug einwirken, um sicher zu gehen, dass die ganze Masse gleichmäßig durchdrungen und nicht einzelne Bakterien der Vernichtung entgangen sind. Durch die Erhitzung allein wird die Fäulnis nicht aufgehoben; die Erfahrung lehrt, dass gekochtes Fleisch, Eier, Milch u. s. w. zwar weit langsamer, aber schließlich ebenso gut faulen wie rohe.

Hat man durch Erhitzung die Bakterien im Glaskölbchen getötet, so muss man noch dafür sorgen, dass nicht neue Keime aus der Luft in das Innere desselben hineingeraten. Für diesen Zweck schmolz im vorigen Jahrhundert ein durch scharfsinnige Experimente berühmter Naturbeobachter, der italienische Abt Spallanzani, den Hals des Kölbchens während des Kochens zu; (Fig. 8.) das Ergebnis war, dass die im Kölbchen eingeschlossenen Tier- und Pflanzenstoffe sich für alle Zeit unverändert hielten ohne jemals zu faulen.

\begin{figure}[H]
\centering
\includegraphics[height=0.1\textheight,,keepaspectratio]{figs/fig08.png}
\caption{}
\end{figure}

\begin{figure}[H]
\centering
\includegraphics[height=0.1\textheight,keepaspectratio]{figs/fig09.png}
\caption{}
\end{figure}

\begin{figure}[H]
\centering
\includegraphics[height=0.1\textheight,keepaspectratio]{figs/fig10.png}
\caption{}
\end{figure}

\paragraph{}
Der französische Graf Appert benutzte am Anfang unseres Jahrhunderts diese Methode, um Fleisch, Gemüse und andere Nahrungsmittel aufzubewahren, indem er dieselben in Blechbüchsen, die mit einer kleinen Öffnung versehen, einschloss, sodann im Wasserbad ein Paar Stunden kochte und während des Kochens die Öffnung zulötete. Jede Hausfrau weiß, dass sich in Blechbüchsen die Speisen Jahrelang halten, ohne zu verderben; die Industrie beschäftigt sich mit dem Einlegen von Nahrungsmitteln im Großen nach dieser Methode; erhalten wir doch durch dieselbe sogar Rindfleisch aus Australien, Hummern aus Amerika, die vielleicht Jahre alt, beim Gebrauch sich wie frische verhielten.

Man hat nun freilich eingewendet: der Grund, dass die in den Spallanzanischen Kölbchen und den Appert'schen Blechbüchsen eingeschlossenen Stoffe nicht faulen, ist nicht der, weil in ihnen keine Bakterien, sondern weil in ihnen kein Sauerstoff anwesend ist; denn es wird ja beim Kochen die Luft ausgetrieben und der Zutritt neuen Sauerstoffs durch das Zulöten unmöglich. Um diesen Einwand zu widerlegen, müsste man in die hermetisch verschlossenen Gefäße Luft zulassen, die keine Bakterien enthält. Zu diesem Zweck änderte Dr. Schwann 1837 den Spallanzani'schen Versuch so ab, dass er den Kolbenhals erst zuschmolz, nachdem in denselben Luft eingetreten, welche durch ein glühendes Rohr gestrichen war; in diesem wurden natürlich alle lebendigen Keime zerstört.

Schröder und Dusch gaben 1854 ein bequemeres Mittel; sie verstopften den offenen Hals des Kölbchens mit gereinigter Watte; indem die Luft in das gekochte Kölbchen beim Abkühlen desselben eindrang, setzte sie zwischen den Fasern der Baumwolle, wie in einem Filter, alle Keime ab. (Fig. 9.)

Endlich ersann Pasteur 1862 ein noch einfacheres Verfahren, er bog den in eine Spitze ausgezogenen Kolbenhals hakenförmig nach unten, ohne ihn zuzuschmelzen; die in der Luft enthaltenen Keime, welche der Schwere folgend, sich gewöhnlich in offenen Gefäßen absetzen, konnten nunmehr nicht ins Innere des Kölbchens gelangen. (Fig. 10.)

Das Ergebnis aller drei Verfahren ist immer das nämliche: die im Kölbchen eingeschlossenen Stoffe geraten niemals in Fäulnis; gleichwohl fehlt es ihnen nicht an Luft; nur die Bakterien finden keinen Eingang. Aus diesen und vielen ähnlichen Versuchen lässt sich mit der größten Sicherheit schließen: dass wo auch alle übrigen Bedingungen der Fäulnis gegeben sind, diese doch nicht stattfindet, wenn keine Bakterien anwesend sind; dagegen beginnt die Fäulnis augenblicklich, sobald Bakterien absichtlich oder unabsichtlich zugesetzt werden, sei es auch in geringster Zahl; die Fäulnis schreitet in demselben Maße fort, in dem sich die Bakterien vermehren; alle Umstände, welche die Vermehrung der Bakterien begünstigen, beschleunigen die Fäulnis; alle Bedingungen, welche deren Entwickelung aufhalten, verlangsamen die Fäulnis; alle Mittel, welche Bakterien töten, heben auch die Fäulnis auf; umgekehrt hört die Vermehrung der Bakterien auf, sobald alle fäulnisfähige Substanz zerstört ist.

Also sind die Bakterien nicht die zufälligen Begleiter, sondern sie sind die Ursache der Fäulnis; Fäulnis ist ein von Bakterien erregter chemischer Prozess. Nicht der Tod, wie man gewöhnlich glaubt, erzeugt die Fäulnis, sondern das Leben jener unsichtbaren Wesen.

Es scheint beinahe selbstverständlich, dass jeder Körper, von dem das Leben gewichen, der Verwesung anheimfällt; und doch steht fest: ohne die Lebenstätigkeit der Bakterien würden alle Geschöpfe auch nach ihrem Tode Form und Mischung beibehalten, so gut wie die ägyptischen Mumien, die in den dänischen Torfmooren versunkenen Recken, oder wie die Mammut- und Rhinozerosleichen, die seit ungezählten Jahrtausenden im sibirischen Eise eingefroren, sich mit Haut und Haar unversehrt erhalten haben. Sobald das Eis schmilzt, verfallen auch diese letzten Überbleibsel einer ausgestorbenen Tierwelt in wenig Tagen der Verwesung: die Ursache ist leicht begreiflich: die Bakterien stellen in der Nähe des Gefrierpunktes ihre Lebenstätigkeit ein, während sie bei etwas höherer Temperatur sich sofort vermehren und Fäulnis erregen. Im Torfmoor und in den Mumien ist es die chemische Mischung, welche die Entwicklung der Bakterien verhindert. Wenn sich in einem nach der Methode von Spallanzani, Schröder und Dusch, oder Pasteur eingerichteten Kölbchen ein Stückchen Fleisch oder ein Pflanzenstoff Jahrelang unverändert erhalten hat, so braucht man nur einen einzigen Bakterienhaltigen Wassertropfen zuzusetzen, um sofort die Fäulnis einzuleiten.

Die gesamte Naturordnung ist darauf gegründet, dass die Körper, in denen das Leben erloschen, der Auflösung anheimfallen, damit ihre Stoffe wieder neuem Leben dienstbar werden können. Denn die Masse des Stoffes, welche sich zu lebenden Wesen gestalten kann, ist auf der Erde beschränkt; immer die nämlichen Stoffteilchen müssen in ewigem Kreislauf von einem abgestorbenen in einen lebenden Körper übergehen; ist auch die Seelenwanderung eine Mythe, so ist die Stoffwanderung eine naturwissenschaftliche Tatsache. Gäbe es aber keine Bakterien, so würden die in einer Generation der Tiere und Pflanzen verkörperten Stoffe auch nach deren Ableben gebunden bleiben, wie es die chemischen Verbindungen in den Felsgesteinen sind; neues Leben könnte sich nicht entwickeln, weil es ihm an Körperstoff fehlen müsste. Indem die Bakterien in rascher Fäulnis jeden abgestorbenen Leib zu Erde werden lassen, machen sie allein das Hervorsprießen neuen Lebens, und damit die Fortdauer der lebendigen Schöpfung möglich.

Die wunderbare Tatsache, dass die Fäulnis eine Arbeitsleistung der Bakterien ist, steht nicht vereinzelt da; es gibt eine ganze Reihe von chemischen Veränderungen, welche durch Bakterien und ähnliche mikroskopische Wesen erregt werden; man bezeichnet diese Vorgänge gewöhnlich als Gährungserscheinungen, und die Wesen, welche die Ursachen derselben sind, als Fermentpilze. Die Bakterien, und zwar die von den Naturforschern als \emph{Bacterium Termo} bezeichnete Art (Fig. 2), sind das Ferment der Fäulnis.

Dasjenige Ferment, welches am längsten bekannt und am genauesten untersucht worden, ist der Alkoholhefepilz (\emph{Sacharomyces cerevisiae}); seine ovalen Kügelchen wurden schon von Leeuwenhoek im Bier beobachtet, aber erst 1837 von Cagniard Latour und fast gleichzeitig von Schwann als die eigentlichen Erreger jener Gärung erkannt, welche den Zucker in Alkohol und Kohlensäure umwandelt, während nebenbei noch kleine Mengen von Glycerin und Bernsteinsäure gebildet werden. Die genauste Erkenntnis über das Verhalten der Hefepilze bei der Alkoholgärung verdanken wir Pasteur, dem wir den Ruhm eines der genialsten und exaktesten Forscher des heutigen Frankreichs nicht schmälern wollen, wenn derselbe sich auch von der Geschmacklosigkeit nicht ferngehalten hat, die Bitterkeit nationaler Gereiztheit auf das neutrale Gebiet der Wissenschaft zu übertragen. Pasteur zeigte, dass der Hefepilz aus denselben Stoffen besteht, wie alle anderen Pflanzen, aus Kohle, Sauerstoff, Wasserstoff, Stickstoff und einer Anzahl Mineralstoffen, unter denen Kali und Phosphorsäure die wichtigsten sind; soll der Hefepilz wachsen und sich vermehren, so muss er diese Stoffe sämtlich als Nahrung empfangen und sie durch seine Lebenstätigkeit zum Bau seiner Zellen verwenden. Der Hefepilz findet die Gesamtheit seiner Nährstoffe nicht in reinem Zucker, wohl aber im ausgepressten Traubensaft, in der Bierwürze und andern gärungsfähigen Flüssigkeiten; er vermehrt sich nur, solange er dieselben findet. Sauerstoff und Wasserstoff werden ihm im Wasser dargeboten; auch die Mineralstoffe müssen in der Lösung vorhanden sein; sie lassen sich später wieder in der Hefeasche nachweisen. Vom Stickstoff glaubte man früher, dass ihn der Hefepilz nur aus den eiweißartigen Verbindungen aufnehmen könne, welche im Traubensafte wie in der Bierwürze nie fehlen; Pasteur zeigte, dass der Hefepilz seinen Stickstoffbedarf auch durch Aufnahme von Ammoniak befriedigen kann, welches aus Wasserstoff und Stickstoff besteht. Die Kohle endlich entnimmt der Hefepilz unmittelbar und ausschließlich aus dem Zucker; er bildet seine Zellhaut und seinen Fettgehalt durch geringe Umwandlungen des Zuckers; vermutlich erzeugt er auch die Eiweißstoffe, die in seinen Zellen vorhanden sind, durch Verbindung des Zucker mit Ammoniak. Indem nun der Hefepilz den Zucker verbraucht, um daraus seine eigenen Zellen zu bilden, zu ernähren und zu vermehren, bewirkt er ein Zerfallen des Zuckers und eine neue Anordnung seiner feinsten Stoffteilchen; er verursacht dadurch eben jene Veränderung, die als Alkoholgärung bezeichnet wird. Ist die Gärung vorüber, so ist der Zucker verschwunden; aber auch der Hefepilz kann sich nun nicht weiter vermehren, er setzt sich am Boden der ausgegorenen Flüssigkeit als Unterhefe ab, oder wird mit der stürmisch entweichenden Kohlensäure als Schaum oder Oberhefe ausgeworfen.

Andere Gärungen werden durch Bakterien oder durch mikroskopische Wesen erregt, die den Bakterien verwandt, nur durch Spaltung oder Teilung ihrer Zellen sich vermehren, und deshalb mit den Bakterien in der Klasse der Spaltpilze (\emph{Schizomyceten}) vereinigt werden. Wenn Bier oder Wein an der Luft mit der Zeit sauer werden, so bildet sich Essigsäure; durch Bakterien, welche in lange Ketten gereiht, oder zu schleimigen Häuten verbunden sind, wird der Alkohol der geistigen Flüssigkeit in Essigsäure verwandelt. Pasteur hat gezeigt, dass alle Krankheiten des Weines von mikroskopischen Fermentpilzen verursacht werden, deren Keime während der Weinbereitung in die Flüssigkeit gelangen und sich darin mehr oder weniger rasch vermehren; ihm gebührt das Verdienst, diese Entdeckung zugleich praktisch zum größten Vorteil des Weinbaues verwertet zu haben; wenn der Wein in den Flaschen auf 50-70° erwärmt wird, so wird nicht bloß das Essigferment, sondern auch die übrigen Spaltpilze getötet, die den Wein kahmig, schleimig, oder bitter machen; der Wein wird haltbar, er kann ausgeführt werden, und gewinnt an Feuer, Bouquet und Werth.

Wenn süße Milch sauer wird, so beruht dies darauf, dass der Milchzucker in Milchsäure verwandelt wird. Auch hier ist ein Fermentpilz aus der Klasse der Bakterien tätig, wie Pasteur zuerst nachgewiesen; wird die Milch gekocht, so wird das Milchsäureferment getötet; und wird der Zutritt neuer Keime verhindert, so hält sich die Milch durch unbegrenzte Zeit süß. Das nämliche Milchsäureferment spielt auch bei der Bereitung des Sauerkrauts, der Sauergurken u. s. w. eine Rolle; entwickelt es sich im Rübensaft oder in der Bierwürze, so macht es den Fabrikanten großen Schaden.

Andere Fermentpilze erzeugen andere Gärungen; eine Art macht den Harn alkalisch, eine andere verwandelt Gerbstoff in Gallussäure, wieder andere sind bei der Buttersäuregärung und bei der Bildung der Käse tätig; besonders interessant sind die Fermentpilze aus der Klasse der Kugelbakterien, welche Farbstoffe erzeugen.

Seit uralter Zeit geht die Sage, dass sich von Zeit zu Zeit auf Speisen, besonders auf Brot, plötzlich ein Bluttropfen bilden könne; ist erst einer erschienen, so vermehrt sich das Blut, es tropft und überzieht weite Flächen; wurde dies in alter Zeit beobachtet, so galt es als ein Unheildrohendes Zeichen, das den Zorn der Gottheit anzeigt, verborgene Verbrechen offenbart und blutige Sühne erheischt, Die Geschichte berichtet bis in die neue Zeit von zahllosen Opfern, welche einem finsteren Aberglauben fielen, so oft das Wunder des Bluts auf Speisen, besonders aber, wenn es auf der geheiligten Oblate einer Hostie sichtbar ward. Mit dem Jahrhundert der Aufklärung hörte allmählich das Blutwunder auf; aber erst seit den letzten Jahrzehnten erkannte man, dass den Wunderberichten eine naturwissenschaftliche Tatsache zu Grunde liege.

Ehrenberg war es, der zuerst die Blut-Erscheinung auf das Sorgfältigste erforschte; sie bildet sich in feuchter Luft, nur auf gekochten, nicht auf rohen Speisen; auf Kartoffeln, Reis, Mehlkleister, Polenta, selbst auf Fleisch, Milch und Hühnereiweiß, von selbst, ohne dass man sie jedoch willkürlich hervorrufen könnte. Zuerst erscheinen meist kleine, rosenrote oder purpurne Schleimtröpfchen, die zur Größe eines starken Stecknadelkopfes anwachsen und wie Fischrogen aussehen, dann sich verflachen, zusammenfließen und einen zähen blutigen Schleim bilden. Breitet man mit der Nadel einen Tropfen der roten Gallert auf einer frischen Kartoffel aus, so vermehrt sich rasch die rote Substanz; es ist leicht, so große Mengen zu erzeugen, dass man sie zum Färben benutzen könnte; leider ist der prächtige Farbstoff nicht haltbar; er wird am Licht bald zerstört. Ehrenberg fand in dem roten Schleim unzählige ovale Körperchen, denen er den Namen der Wundermonaden (\emph{Monas prodigiosa}) gab; wir bezeichnen sie besser als rote Kugelbakterien (\emph{Micrococcus prodigiosus}) (Fig. 1); sie ernähren sich von den eiweißhaltigen Speisen, auf deren Oberfläche sie sich entwickeln, zersetzen dieselben und erzeugen durch eine eigentümliche Pigmentgärung den roten Farbstoff, der, wie Otto Erdmann\footnote{Bildung von Anilinfarben aus Proteinkörpern. Journal für praktische Chemie. 1866.} und Schroeter\footnote{Schröter über einige durch Bakterien gebildete Pigmente in Cohns Beiträgen zur Biologie der Pflanzen. Heft 2. 1872.} nachgewiesen haben, eine auffallende Verwandtschaft mit jenen glänzenden Anilinfarben besitzt, welche in der neuesten Zeit eine so hohe Bedeutung für die Färbeindustrie gewonnen haben.

An historischem Interesse, und dem mächtigen Eindruck, welchen es auf die mythenbildende Phantasie der Völker ausübte, steht das "`Wunderblut"' einzig da; als naturwissenschaftliche Erscheinung schließt es sich an eine ganze Reihe von Färbungen, welche in feuchter Luft fast regelmäßig auf Kartoffeln, auf Käse, gekochten Eiern und anderen Speisen erscheinen, in Gestalt schneeweißer, schwefelgelber, orangeroter, spangrüner, violetter, blauer oder brauner Flecken, Tröpfchen und Schleimmassen; alle diese Farben, zum Teil ebenfalls Anilinpigmenten verwandt, werden von Kugelbakterien erzeugt, welche unter dem Mikroskope sich von dem \emph{Micrococcus prodigiosus} des Wunderbluts kaum unterscheiden lassen. Wenn sich die Milch von selbst blau oder gelb färbt, oder der Eiter aus Wunden eine spangrüne Färbung annimmt, so sind Stäbchenbakterien als Erzeuger der Farbstoffe in diesen Flüssigkeiten nachgewiesen.\footnote{\emph{Bacterium (Vibrio) synxanthum, Bact. syncyanum Ehr., Bact. aerugineum Schoet.}} Der von den Chemikern so viel benutzte Lakmus wird nebst einigen verwandten Pigmenten aus strauchigen oder krustigen Felsen-bewohnenden Flechten gewonnen, indem dieselben im Wasser so lange der Fäulnis überlassen bleiben, bis der anfänglich farblose Auszug an der Luft eine schöne purpurne, rote oder blaue Färbung annimmt; nach neueren Forschungen ist es wahrscheinlich, dass auch der Lakmus durch die Lebenstätigkeit von Bakterien gebildet wird; es ist sogar gelungen, durch Kugelbakterien in künstlichen chemischen Lösungen, welche an sich wasserklar und vollkommen farblos, eine gewisse Menge weinstein- und essigsaures Ammoniak enthalten, in kurzer Zeit einen dem Lakmus ganz ähnlichen blauen Farbstoff zu erzeugen, der die Flüssigkeit erst hellblau, von Tag zu Tag immer prächtiger und tiefer blau färbt; in andern Versuchen traten Kugelbakterien gewissermaßen als Fabrikanten von saft- oder spangrünen, gelben oder roten Farben auf, die sie aus farblosen chemischen Lösungen herzustellen vermögen.

Endlich hat sich in jüngster Zeit ein ungeahnter Einblick in geheimnisvolle Lebenstätigkeiten der Bakterien eröffnet, durch welche dieselben mit dämonischer Gewalt über Wohl und Wehe, ja über Leben und Sterben der Menschen entscheiden.

Häufiger vielleicht als je in Folge des gesteigerten Völkerverkehrs, sind in den letzten Jahrzehnten Menschen und Tiere von der Gottesgeißel der Epidemien heimgesucht worden, die mit unaufhaltsamem Schritt von Stadt zu Stadt, von Land zu Land wandern, einen einzelnen Ort nur eine Zeit lang heimsuchen, dann gleichsam ermattend verschwinden, um an einer anderen Stelle ihr Werk fortzusetzen, und meist erst nach längerer Zwischenzeit wieder zurückzukehren. Nur zu oft vergeblich bemüht sich ärztliche Kunst und Wissenschaft, der verheerenden Gewalt dieser Krankheiten ihre Opfer zu entreißen, oder ihrem Gange durch Vorbeugungsmaßregeln Schranken zu setzen. So verschieden auch die einzelnen Krankheitsbilder, so haben doch alle Epidemien, Cholera, Pest, Typhus, Diphtherie, Pocken, Scharlach, Hospitalbrand, Rinderpest und wie sie alle heißen, gewisse gemeinschaftliche Züge: die Krankheit entsteht nirgends von selbst, weder aus äußeren noch aus inneren Ursachen; sondern sie wird aus einem anderen Ort eingeschleppt, wo sie bereits früher herrschte, durch einen Kranken, oder durch Gegenstände, die mit einem Kranken in Berührung waren; sie verbreitet sich nur durch Ansteckung. Hat die Ansteckung stattgefunden, so vergehen Stunden und selbst Tage, ehe die Zeichen derselben äußerlich hervortreten; nach einer gewissen Zeit, der Inkubation, bricht die Krankheit aus durch gewaltsame Störungen in derkgesetzmäßigen Lebenstätigkeit aller Organe, vom Gehirn bis zum Verdauungssystem; der Kranke leidet, als stände er unter dem Einfluss eines Giftes, welches in sein Blut eingedrungen; und wie er selbst durch einen Giftstoff angesteckt, so verbreitet er wieder das Gift weiter, im Athem, im Schweiß, in den Ausleerungen, selbst in den Kleidern, oder der Wäsche; in manchen Krankheiten sammelt sich der Ansteckungsstoff in konzentriertester Form in besonderen Pusteln oder Blattern, deren klarer Saft schon in der geringsten Menge einen Gefunden vergiftet, sobald er in dessen Blutlauf aufgenommen wurde, und ihn unter den nämlichen Krankheitserscheinungen zum Erzeuger des nämlichen Giftes werden lässt. Beim Hospitalbrand, beim Leichengift genügt schon der Hauch, der am Messer des Chirurgen oder des Anatomen haftet, um jede offene Wunde zu vergiften: beim Milzbrand steht fest, dass eine Fliege das Gift von einem Kranken auf ein gesundes Tier übertragen kann.

Kaum hatte Leeuwenhoek seine ersten Beobachtungen über die unsichtbaren Tierchen im Regenwasser bekannt gemacht, als die vorschnelle Hypothese phantastischer Ärzte das furchtbare Rätsel der Epidemien durch mikroskopische Pestfliegen zu erklären glaubte. Aber vergeblich blieb bis in die neueste Zeit jeder Versuch, in dem Ansteckungsstoff, welcher durch Berührung die Krankheit erzeugt, oder in dem Kontagium mit Hülfe des Mikroskops lebende Wesen wirklich aufzufinden; es wäre ebenso leicht gewesen, die unsichtbaren Pfeile zu Gesicht zu bekommen, mit denen nach dem Glauben der Alten der ferntreffende Apollon in seinem Zorn Menschen und Heerden hinstreckte.

Die erste Entdeckung mikroskopischer Organismen in einer ansteckenden Krankheit verdanken wir Davaine, welcher im Jahre 1863 im Blute milzkranker Rinder einige Stunden vor deren Tode unzählige feine fadenförmige Körperchen beobachtete, die meist doppelt so lang als Blutkörperchen, sich durch Teilung vermehren und von den gewöhnlichen Fadenbakterien sich nur durch den Mangel an Bewegung unterscheiden; Davaine bezeichnete sie deshalb als Bakteridien. Auch der Mensch ist einer ansteckenden Krankheit unterworfen, die dem Milzbrand sehr nahe verwandt ist\footnote{\emph{Pustula maligna.}}; auch in diesen Fällen ist sein Blut von Bakteridien erfüllt.

Seit etwa vier Jahren hat sich die Zahl der Epidemien, bei denen Bakterien auftreten, sehr vermehrt\footnote{Wir verdanken diese Tatsachen den Untersuchungen von Keber, Hallier, Zürn, Klebs, Leiden, Recklingshausen, Jaffe, Waldeyer, Orth, Buhl, Hüter, Oertel, Traube und Anderen.}: es ist jedoch hier nicht am Orte die einzelnen Fälle zu besprechen; wir greifen nur einige der wichtigsten, am genauesten untersuchten Vorkommnisse heraus.

Jedermann weiß, wie erbarmungslos die Diphtherie so manches hoffnungsvolle Leben hinwegrafft; ein leicht übertragbares Kontagium setzt sich gewöhnlich zuerst in Schlund und Luftröhre fest, erzeugt dort membranartige Gebilde, welche mit raschem Erstickungstod bedrohen. Das Mikroskop zeigt in sämtlichen Organen des Kranken unzählige Kugelbakterien in dichten Massen zusammengehäuft, welche die Gewebe der Muskeln, Gefäße, Schleimhäute durchsetzen und belagern, überall Blutstauungen und Entzündungen herbeiführen und eine allgemeine Blutvergiftung zur Folge haben. Nur dann ist Genesung möglich, wenn die Kugelbakterien in den Nieren sich anhäufen und durch diese allmählich aus dem kranken Körper wieder ausgeschieden werden.

Die Blutvergiftung durch offene Wunden, welche im Kriege mehr Opfer wegrafft, als die feindlichen Kugeln, und wenn sie einmal in einem Hospital sich eingenistet, selbst leichte Verletzungen tödlich werden lässt, ist stets von der Vermehrung von Kugelbakterien begleitet, die bald vereinzelt, bald in rosenkranzförmigen Fäden oder in schleimigen Haufen sich im Eiter und im Narbengewebe ansiedeln, oder ins Blut aufgenommen und in verschiedenen Organen abgesetzt werden, wo sie Entzündung, Eiterung, Abszessbildung herbeiführen, und durch zehrende Fieber die jugendlichste Lebenskraft erschöpfen. Auch in der klaren Lymphe der Kuh- und Menschenpocken sind ähnliche Kugelbakterien in ungeheurer Menge und rascher Vermehrung aufgefunden worden. In den Ausleerungen der Cholerakranken, welche mit Reiswasser verglichen werden, hat Klob schon im Jahre 1866 unzählige Bakterien, zu gallertartigen Schleimmassen verbunden, nachgewiesen. Selbst die Seidenwürmer unterliegen einer Epidemie, bei der Bakterien auftreten.

Aber folgt denn aus der Gegenwart der Bakterien, dass dieselben auch wirklich mit der Epidemie zu schaffen haben? Ist es nicht eben so gut möglich, dass diese mikroskopischen Wesen nur zufällige und unwesentliche Begleiter der Krankheit sind, wie ja Bakterien sich bei jeder Gärung und Fäulnis entwickeln, ohne den mindesten Einfluss auf die Gesundheit auszuüben?

Noch ist das durch die neuesten Forschungen verbreitete Licht nicht hell genug, um dieses dunkle Gebiet ganz überschauen zu lassen; noch ist der neu gewonnene Boden nicht so fest, um das Gebäude einer unerschütterlichen Theorie darauf zu gründen. Doch das wissen wir bereits, dass die Bakterien der Kontagien nicht die nämlichen Arten sind, welche Fäulnis erregen; sie lassen sich von den letzteren meist schon unter dem Mikroskop durch ihre Form unterscheiden; sie stehen unter ganz anderen Lebensbedingungen; ja sie kämpfen oft mit den Fäulnisbakterien auf dem nämlichen Boden um das Dasein und werden von diesen ausgerottet, wenn sie unterliegen. Das hatte schon Davaine gefunden, als er beobachtete, dass mit beginnender Fäulnis, oft schon 48 Stunden nach dem Tode eines Tiers, die Milzbrandbakterien verschwinden, sobald die gemeinen Stäbchenbakterien sich maßlos vermehren. Während aber ein Blutstropfen voll Milzbrandbakterien einem gesunden Rinde eingeimpft, nach 24 bis 36 Stunden den Tod bringt, so ist die Impfung mit gefaultem Blute ohne Bakteridien wirkungslos. Durch Eintrocknen verlieren die Milzbrandbakterien ihre Lebensfähigkeit nicht; daher gelingt auch die Ansteckung durch getrocknetes Blut.

Bekanntlich gehen durch ein dichtes Filter, einen Thonzylinder, oder durch eine Membran nur klare Flüssigkeiten; feste Körperchen und wären sie noch so klein, werden vom Filter zurückgehalten. Diese Erfahrung benutzten Chauveau und Klebs, um zu beweisen, dass bei Pyämie, Septikämie und Blattern das Kontagium nicht in den flüssigen Teilen des Eiters oder der Lymphe seinen Sitz haben könne, sondern in den mikroskopischen Kugelbakterien, welche sich darin entwickeln. Indem sie nämlich diese Ansteckungsstoffe durch ein Filter seihten, ermittelten sie, dass die klare Flüssigkeit, welche durch das Filter gegangen, ihre Ansteckungsfähigkeit verloren hatte, während die auf dem Filter zurückgebliebenen festen Substanzen wirksam blieben.

Alle diese Tatsachen machen es in hohem Grade wahrscheinlich, dass die in vielen Krankheiten bereits nachgewiesenen Bakterien die Träger und Erreger der Ansteckung, dass sie die Fermente der Kontagien sind. Wir halten an der Hoffnung fest, dass sich an eine vollständigere und klarere Erkenntnis dieser Tatsachen auch die Auffindung neuer Methoden knüpfen wird, um dem furchtbaren Feinde mit besserem Erfolge als bisher entgegenzutreten. Der Kunst des Arztes würden dadurch bestimmte Gesichtspunkte gegeben, auf welche sie hinzuwirken hat; es handelt sich um die drei Fragen: auf welchem Wege geschieht und auf welche Weise verhindert man die Übertragung von mikroskopischen Fermentorganismen? und durch welche Mittel wird die Vermehrung derselben gehemmt? Alle Desinfektionsmaßregeln, alle Heilversuche müssten nach der einen oder der anderen Richtung hineingreifen; besonders würde auch das Wasser ins Auge zu fassen sein, von dem festgestellt ist, dass es selbst in scheinbar reinstem Zustande doch die Zufuhr von Bakterien und andern Fermentorganismen leicht vermittelt.

Wir haben gesehen, dass bei aller Fäulnis und Gärung, dass in vielen Krankheiten sich Bakterien entwickeln und in riesigen Verhältnissen vermehren, sobald ihre Keime einmal Zugang gefunden, dass diese kleinsten Wesen gerade durch ihre Massenentwicklung die großartigste Arbeit verrichten. Aber woher stammen die ersten Keime? Mit dieser Frage haben sich die Naturforscher bis in die neueste Zeit beschäftigt, und sie in verschiedenem Sinne beantwortet.

Die Einen sagten: bei der Fäulnis formen sich die organischen Elemente, welche den Körper des abgestorbenen Tiers gebildet hatten, in freier Schöpfungskraft zu selbständigen Wesen, die ganz verschieden von denen, aus deren Stoffen sie hervorgegangen, doch ebenfalls belebt und fortpflanzungsfähig sind; so gestalten sich die Eiweiß- und Fetttröpfchen zu Bakterien, vielleicht auch zu Hefe- und Schimmelpilzen, selbst zu jenen Infusionstierchen, die bei der Verwesung nie fehlen. Man erfand sogar für diese Weise der Entstehung einen besondern Namen, Urzeugung (\emph{Generatio aequivoca}).

Die Andern bestreiten die Möglichkeit dass lebende Wesen, seien sie noch so klein und einfach, jemals anders entstehen als aus Keimen, die von Wesen gleicher Art abstammen. Der Glaube an die Urzeugung der Bakterien sei der letzte Überrest eines uralten Aberglaubens, den die Leuchte der Wissenschaft noch nicht ganz verscheucht hat. Im Altertum meinte man, Schlangen und Frösche entständen aus dem Schlamm, den die Sonne bebrütet, Raupen erzeugten sich aus faulen Blättern, Ungeziefer aus Schmutz, Würmer aus kranken Eingeweiden, Maden aus verdorbenem Fleisch. Heutzutage weiß jedes Kind, dass alles dies Märchen sind; jede Hausfrau hat die Erfahrung gemacht, dass im Fleisch keine Maden entstehen, wenn durch ein Drahtgitter den Schmeißfliegen der Zutritt verwehrt wird, die ihre Eier darin ablegen wollen; sie hat gelernt, durch sorgfältiges Bedecken die staubfeinen Schimmelsporen abzuhalten, welche mit anderem Staube aus der Luft abgesetzt, auf ihren eingelegten Früchten gern sich ansiedeln; sie weiß, dass Trichinen und Bandwürmer nur durch den Genuss von rohem oder halbgekochtem Schweinefleisch entstehen, in dem die Jugendzustände dieser Tiere bereits vorhanden waren; selbst die Landwirte glauben nicht mehr, dass der Getreiderost durch Erkältung erzeugt wird, sondern dass er von Keimen abstammt, die von Berberizensträuchern oder von andern befallenen Halmen ausgestreut werden, und dass der Brand im Weizen verhindert wird, wenn man das Saatgut in Kupfervitriol einbeizt, um die anhaftenden Sporen des Brandpilzes zu töten.

Für die Bakterien und die ihnen verwandten Fermentpilze ist durch die von uns schon oben erwähnten Versuche der zweifellose Beweis geführt, dass sie eben so wenig durch Urzeugung entstehen, als andre lebende Wesen. Denn wenn Fleisch oder ein andrer stickstoffhaltiger Stoff aus dem Tier- oder Pflanzenreich in einem Kölbchen gekocht, ja auch nur auf ca. 60° erhitzt wird, so werden alle darin vorhandenen Bakterien getötet; wird nun der Zutritt neuer Keime von Außen auf die eine oder die andre Weise verhindert, so entstehen nie und nimmer Bakterien von selbst, möge man das Kölbchen auch noch so lange aufbewahren; ein einziger eingeführter Keim dagegen genügt, um die Vermehrung und mit dieser die Fäulnis zu veranlassen. Entständen die Bakterien aus faulenden Stoffen durch Urzeugung, so müsste die Fäulnis dem Erscheinen der Bakterien vorangehen; der Versuch aber zeigt das Gegenteil, dass die Fäulnis erst eine Folge der Bakterienentwickelung ist.

In den letzten Jahren machte eine Theorie großes Aufsehen, welche die Entstehung der Bakterien auf andre Weise zu erklären suchten. Die gewöhnlichen Schimmelpilze sollten unter gewissen Bedingungen bewegliche Keime von außerordentlicher Kleinheit gebären; diese Keime können sich, wurde behauptet, zu Bakterien, zu Hefe, schließlich wieder zu Schimmelpilzen fortentwickeln. Wenn sich in gewissen Krankheiten Bakterien im Blut oder in andern Organen finden, so beruhe dies darauf, dass die Sporen gemeiner Schimmel- oder Brandpilze im menschlichen Körper keimen, dass diese Keime erst als Bakterien schwärmen, sich aber bei geeigneter Kultur wieder zu verschiedenen Arten von Schimmelpilzen erziehen lassen. Aber eine vorurteilsfreie Nachprüfung hat nicht den geringsten Beweis dafür gegeben, dass Bakterien mit Hefe, Brand- oder Schimmelpilzen in entwicklungsgeschichtlichem Zusammenhang stehen; die Bakterien entstehen, so viel wir bis jetzt wissen, immer nur aus Keimen gleicher Art.

Durch diese Tatsachen ist freilich die Hoffnung zu Nichte gemacht worden, dass in der Entwickelung der Bakterien der Schlüssel gefunden werde für den Ursprung des Lebens auf der Erde überhaupt. Gäbe es auch nur ein einziges Wesen, welches aus ungeformter und lebloser Materie sich von selbst durch Urzeugung noch heutzutage zu einer lebendigen Zelle gestalten kann, so könnten wir uns vorstellen, dass die ersten Geschöpfe sich am Anfang auf die nämliche Weise gebildet haben. Nunmehr steht zwar fest, dass das Leben auf Erden einen Anfang gehabt; wie aber die ersten lebendigen Wesen entstanden, dafür fehlt es an aller Analogie; nach unserem bisherigen Wissen gleicht das Leben dem heiligen Feuer der Vesta, welches dadurch ewig erhalten wurde, dass immer der neue Brand sich an dem alten entzündete.

Der berühmte Physiker W. Thomson hat in der geistvollen Rede, mit welcher er im vorigen Jahre die britische Naturforscherversammlung zu Edinburgh eröffnete, die Schlussfolgerung gezogen: da das Leben auf der Erde nicht von selbst entstanden sein könne, so müsse es von einem andern Weltkörper auf den unsrigen übertragen worden sein. Wir wissen, dass die unzähligen Meteorsteine, welche auf die Erde herabgefallen sind, einst selbstständige Weltkörper oder doch Teile von solchen gewesen; in einzelnen Meteoriten sind Kohle und kohlenhaltige Verbindungen nachgewiesen, deren Ursprung auf organische Bildung hindeutet. Es lässt sich die Möglichkeit denken, dass auch einmal ein lebender und entwickelungsfähiger Keim die Katastrophe überlebt habe, welche gewöhnlich den Ankömmling aus dem Weltraum beim Eintritt in unsere Atmosphäre und beim Herabsturz auf die Erde in Gluth versetzt; von solchem Keime mögen alle andern Wesen abstammen; so mag auf die lebensleere Erde einstmals der Anfang des Lebens vom Himmel herabgekommen sein, wie nach der Mythe der belebende Feuerfunke durch Prometheus vom Olymp geholt wurde.

Die Entwickelungsgeschichte der Bakterien lässt vielleicht an einen andern Ursprung des Lebens auf der Erde denken. Wir haben das Gewicht einer Bakterie auf 0,00000000157 Mgrm. berechnet; wir wissen, dass diese unendlich leichten Körperchen bei der Verdunstung durch die verdampfenden Wasserteilchen mit fortgeführt, in der Luft als Sonnenstäubchen umherschwimmen, und mit dem Staube wieder herabfallen, aber auch durch Luftströmungen über unermessliche Strecken geführt, und gewiss auch in außerordentliche Höhe getragen werden können. Möglicherweise werden diese Stäubchen durch aufsteigende Luftströme mitunter so weit emporgehoben, dass sie der Anziehung unseres Planeten entzogen, in den Weltraum gelangen; die Existenz eines Weltstaubes ist aus verschiedenen kosmischen Lichterscheinungen wahrscheinlich. Der Weltraum ist außerordentlich kalt; doch haben Versuche erwiesen, dass selbst ein vielstündiges Einfrieren bei -18° die Bakterien nicht tötet; sie verfallen durch die Kälte in Erstarrung, aus der sie beim Auftauen erwachen und unter günstigen Umständen sich sofort zu vermehren beginnen. Es ist vielleicht nicht unmöglich, dass ein von der Erde aufgestiegenes Bakterienstäubchen eine Zeit lang im Weltraum umherschwimmt, dann in die Atmosphäre eines anderen Weltkörpers gelangt, und wenn es auf diesem die geeigneten Lebensbedingungen vorfindet, dort sich weiter vermehrt. Es lässt sich aber auch umgekehrt die Möglichkeit denken, dass aus irgendeinem Leben ernährenden Weltkörper die Keime einer Bakterie oder eines ähnlichen äußerst kleinen und einfachen Wesens als Stäubchen in den Weltraum geführt werden, und dass ein solcher Keim schließlich in die Atmosphäre der Erde gelangt und auf deren Boden sich absetzt. So lange das Urmeer, welches einstmals die ausglühendem Zustande erstarrte Erdrinde bedeckt hatte, noch über 60° erhitzt war, so lange war eine Entwickelung eines solchen Keimes nicht möglich; so bald aber die Abkühlung unter diesen Temperaturgrad gesunken war, musste der fremde Lebenskeim in dem mit Salzen reich gesättigten Urmeer alle Bedingungen zu einer unbegrenzten Vermehrung finden; wir haben berechnet, dass in wenig Tagen der ganze Ocean mit solchen Wesen erfüllt sein könnte. Aus diesem ersten lebendigen Keim, in dem die Eigentümlichkeiten des Tier- und Pflanzenreichs noch nicht geschieden waren, konnte das Gesetz der Entwickelung, der Kampf ums Dasein, die natürliche Züchtung, die geographische Isolierung und manche andre bekannte oder unbekannte Kraft alle die verschiedenen Formen der Tier- und Pflanzenwelt fortbilden, welche in der Vergangenheit wie in der Gegenwart die Erde bewohnten und bewohnen.

Wir wissen wohl, dass wir mit solchen Betrachtungen weit über die Grenzen der exakten Naturwissenschaft hinausschweifen. Wenn der Naturforscher auch sich der Beschränktheit seines Wissens stets bewusst bleibt und mit Resignation sein Nichtwissen eingesteht, wo seine Werkzeuge, Versuch und Beobachtung, ihn im Stich lassen, so kann er doch nicht immer der Sehnsucht des Faust widerstehen "`zu schauen alle Wirkungskraft und Samen,"' und er überlässt sich gern der Verlockung, durch die Phantasie die Lücken zu ergänzen, welche die nüchterne Forschung nicht auszufüllen vermag.
\end{document}
